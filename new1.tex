\documentclass[a4paper,11pt]{article}
\usepackage{amsmath}
% define the title
\author{Vitesh Shah}
\title{Dislocation Interactions and their formation of subgrains}
\begin{document}
% generates the title
\maketitle
% insert the table of contents
\section{Equations for formation of cell/subgrain boundary} \fussy

\par Nucleation of recrystallization depends on the formation and evolution of sub-grains. The process can be thought as follows: the dislocations form and start migrating across the structure. They encounter each other and get entangled, or get annihilated or form dipoles. Entanglement is the start of formation of subgrain/cell boundary. Moreover, as time proceeds the dislocations in the subgrain boundaries annihilate and re-arrange themselves, leading to a sharper LAGB. At the same time, in the case of dynamic recovery or recrystallization, the newly generated dislocations would interact with the cell/subgrain boundaries. Conversely, as the sub-grains grow the LAGB will interact with the dislocations present in the interior of the sub-grain. All these events must be taken into account in form of equations to enable the evolution of sub-grain size as a function of dislocation density. \par

To account for all these processes, the dislocation densities are distinguished based on different types. 
Here, the dislocation densities are considered as mobile and immobile dislocations. 
Further choice needs to be made, whether further distinguishing of immobile dislocations is required or not. 
If the only immobile dislocations are considered with no further differentiation, some fitting parameters would have to be assumed to account for subgrain/cell size and misorientations. 
The evolution of these different fractions cannot be tracked by physical laws properly. 
Therefore, further differentiation of immobile dislocations into immobile dislocations in the interior and walls, helps in tracking these dislocation densities with different physics based laws. 
This can help in better quantification of subgrain/cell size and misorientations. What??

The Orowan equation gives the relation between the mobile dislocation densities and the shear rate and deformation gradient,

\begin{equation} 
%generates equations
\dot{\gamma}_m^\alpha = \rho_m^\alpha b v^\alpha \label{eq:1}
\end{equation}

where \begin{math} \dot{\gamma}_m^\alpha \end{math} is the shear rate in a particular slip system, which is related to deformation gradient,\begin{math} F_p \end{math}, as \begin{math} \dot{F_p} F_p^{-1} = \sum\limits_{\alpha = 1}^N \dot{\gamma}_m^\alpha m^\alpha \bigotimes n^\alpha \end{math}
, where \begin{math} m^\alpha \end{math} is the slip direction and \begin{math} n^\alpha \end{math} is the slip plane normal. \begin{math} b \end{math} is the burgers vector and \begin{math} v^\alpha \end{math} is the dislocation glide velocity,

\begin{equation} 
%generates equations
v^\alpha = \lambda^\alpha v_o exp\left(\frac{-Q}{{k_B}T}\right) \label{eq:2}
\end{equation}

where \begin{math} \lambda^\alpha \end{math} is the jump width, \begin{math} v_o \end{math} is the attack frequency, \begin{math} Q \end{math} is the activation energy for dislocation glide, 
\begin{math} V \end{math} is the activation volume (defined as  \begin{math} V_\alpha = c_2 b^2 \lambda_\alpha \end{math}), \begin{math} k_B \end{math} is the Boltzmann constant. Different models describe the
\begin{math} \tau_eff \end{math} (frictional stress or passing stress) in different ways. 

The models based on composite structure models assume fractions of different dislocation structures and average the response. This approach has been used by Roters et al. in \cite{Roters2000}, \cite{Ma2004}
Some other authors believe that the size of the dislocation structures (in this case sub-grain and cell structures) also plays a role in determination of the friction stress \cite{Nes1997}. 
As in this current formulation, the volume fraction of the dislocation substructures is not known, formulating the frictional stress based on the dislocation structure size seems to be a better approach.
The cell/subgrain size would determine, how frequently a moving dislocation would encounter some resistance. In addition to this effect, the amount of dislocations in the boundary of the cell/subgrain will also determine 
the resistance to dislocation glide. By considering all these factors together, the frictional/passing stress is considered as following,

\begin{equation} 
%generates equations
\tau_{\text{eff}}^\alpha = \left|\tau^\alpha\right| - c_1 G b \sqrt{\rho_P^\alpha + \rho_m^\alpha} - \frac{c_2 G b}{d_s} - c_3 G b \sqrt{\rho_{\text{imm,wall}}^\alpha} \label{eq:3}
\end{equation}

The third and fourth term in equation \eqref{eq:3} are considered to account for the stress fields of the dislocations present in the cell-walls/subgrain boundaries. 
Smaller cell sizes would lead to mobile dislocations encountering the dense dislocation networks in the cell boundaries more frequently. 
Moreover, the amount of dislocations in the boundary would also contribute to the resistance to the glide of mobile dislocations. Denser dislocation walls will provide higher resistance to dislocation glide
Therefore, the fourth term has been included in the equation. 
This addition of fourth term can help in accounting for effect of reduction or saturation of flow stress due to sharpening of cell/subgrain boundaries, where the growth of cell/sub-grain could be negligible. 

In the initial time step, the dislocations are generated on a particular slip plane and start to glide along it. 
They encounter the forest dislocations and get locked on interaction. 
This interaction is the first step towards formation of a cell boundary. 
Therefore, it can be assumed that the dislocation mean free path on that particular slip plane would determine the radius of the dislocation cell/sub-grain. 
It is further assumed that no other types of dislocations (e.g. Immobile dislocations) are present at the first time step. 
The mean free path is dependent on the forest dislocation density. The parallel and forest dislocations on a particular slip plane can be described as in \cite{Roters2011}
\ldots{} and here it ends.
\bibliographystyle{ieeetran}
%can use bibliographystyle{plain} for the most basic things but it doesnt order the citations properly
\bibliography{library}
\end{document}