\documentclass[a4paper,11pt]{article}
\usepackage{amsmath}
% define the title
\author{Vitesh Shah}
\title{Dislocation Interactions and their formation of subgrains}
\begin{document}
% generates the title
\maketitle
% insert the table of contents
\section{Equations for formation of cell/subgrain boundary} \fussy

\par Nucleation of recrystallization depends on the formation and evolution of sub-grains. The process can be thought as follows: the dislocations form and start migrating across the structure. They encounter each other and get entangled, or get annihilated or form dipoles. Entanglement is the start of formation of subgrain/cell boundary. Moreover, as time proceeds the dislocations in the subgrain boundaries annihilate and re-arrange themselves, leading to a sharper LAGB. At the same time, in the case of dynamic recovery or recrystallization, the newly generated dislocations would interact with the cell/subgrain boundaries. Conversely, as the sub-grains grow the LAGB will interact with the dislocations present in the interior of the sub-grain. All these events must be taken into account in form of equations to enable the evolution of sub-grain size as a function of dislocation density. \par

To account for all these processes, the dislocation densities are distinguished based on different types. 
Here, the dislocation densities are considered as mobile and immobile dislocations. 
Further choice needs to be made, whether further distinguishing of immobile dislocations is required or not. 
If the only immobile dislocations are considered with no further differentiation, some fitting parameters would have to be assumed to account for subgrain/cell size and misorientations. 
The evolution of these different fractions cannot be tracked by physical laws properly. 
Therefore, further differentiation of immobile dislocations into immobile dislocations in the interior and walls, helps in tracking these dislocation densities with different physics based laws. 
This can help in better quantification of subgrain/cell size and misorientations. 

The Orowan equation gives the relation between the mobile dislocation densities and the shear rate and deformation gradient,

\begin{equation} 
%generates equations
\dot{\gamma}_m^\alpha = \rho_m^\alpha b v^\alpha \label{eq:1}
\end{equation}

where \begin{math} \dot{\gamma}_m^\alpha \end{math} is the shear rate in a particular slip system, which is related to deformation gradient,\begin{math} F_p \end{math}, as \begin{math} \dot{F_p} F_p^{-1} = \sum\limits_{\alpha = 1}^N \dot{\gamma}_m^\alpha m^\alpha \bigotimes n^\alpha \end{math}
, where \begin{math} m^\alpha \end{math} is the slip direction and \begin{math} n^\alpha \end{math} is the slip plane normal. \begin{math} b \end{math} is the burgers vector and \begin{math} v^\alpha \end{math} is the dislocation glide velocity,

\begin{equation} 
%generates equations
v^\alpha = \lambda^\alpha v_o exp\left(\frac{-Q}{{k_B}T}\right) \label{eq:2}
\end{equation}

where \begin{math} \lambda^\alpha \end{math} is the jump width, \begin{math} v_o \end{math} is the attack frequency, \begin{math} Q \end{math} is the activation energy for dislocation glide, 
\begin{math} V \end{math} is the activation volume (defined as  \begin{math} V_\alpha = c_2 b^2 \lambda_\alpha \end{math}), \begin{math} k_B \end{math} is the Boltzmann constant. Different models describe the
\begin{math} \tau_eff \end{math} (frictional stress or passing stress) in different ways. 

The models based on composite structure models assume fractions of different dislocation structures and average the response. This approach has been used by Roters et al. in \cite{Roters2000}, \cite{Ma2004}
Some other authors believe that the size of the dislocation structures (in this case sub-grain and cell structures) also plays a role in determination of the friction stress \cite{Nes1997}. 
As in this current formulation, the volume fraction of the dislocation substructures is not known, formulating the frictional stress based on the dislocation structure size seems to be a better approach.
The cell/subgrain size would determine, how frequently a moving dislocation would encounter some resistance. In addition to this effect, the amount of dislocations in the boundary of the cell/subgrain will also determine 
the resistance to dislocation glide. By considering all these factors together, the frictional/passing stress is considered as following,

\begin{equation} 
%generates equations
\tau_{\text{eff}}^\alpha = \left|\tau^\alpha\right| - c_1 G b \sqrt{\rho_P^\alpha + \rho_m^\alpha} - \frac{c_2 G b}{d_s} - c_3 G b \sqrt{\rho_{\text{imm,wall}}^\alpha} \label{eq:3}
\end{equation}

In the equation above, the second term relates to the resistance offered by the other dislocations in the path of a mobile dislocations. 
The third and fourth term in the equation \eqref{eq:3} are considered to account for the stress fields of the dislocations present in the cell-walls/subgrain boundaries.
Smaller cell sizes would lead to the cases where the mobile dislocations encounter the dense dislocation networks in the cell boundaries more frequently. 
Moreover, the amount of dislocations in the boundary would also contribute to the resistance to the glide of mobile dislocations. 
Therefore, the fourth term has been included in the equation. 
This addition of fourth term can help in accounting for effect of reduction or saturation of flow stress due to sharpening of cell/subgrain boundaries, where the growth of cell/sub-grain could be negligible. 
The constants \begin{math} c_1,c_2,c_3 \end{math} have to be currently considered as fitting parameters. 
It can be imagined that \begin{math} c_2 \end{math} would be some function of misorientation of the boundary. 
Sub-grain boundaries with higher misorientations would mean that the mobile dislocation would have to change the slip system a little bit (for sub grain boundaries), which would require extra energy.  
The dislocations in the walls are much more densely packed, which would result in higher resistance to gliding dislocations, for the same order of dislocation densities. 
Therefore, the constant \begin{math} c_3 > c_1 \end{math}

In the initial time step, the dislocations are generated on a particular slip plane and start to glide along it. 
They encounter the forest dislocations and get locked on interaction. 
This interaction is the first step towards formation of a cell boundary. 
Therefore, it can be assumed that the dislocation mean free path on that particular slip plane would determine the radius of the dislocation cell/sub-grain. 
It is further assumed that no other types of dislocations (e.g. Immobile dislocations) are present at the first time step. 
The mean free path is dependent on the forest dislocation density. The parallel and forest dislocations on a particular slip plane can be described as in \cite{Roters2011}
%
\begin{subequations}
\begin{equation}
\rho_F^\alpha = \sum\limits_{\beta = 1}^{N_\text{slip}} \chi^{\alpha\beta} \rho_\text{SSD}^\beta \left|cos \left( n^\alpha , t^\beta \right) \right|  \label{eq:4.1} \\
\end{equation}
\begin{equation}
\rho_P^\alpha = \sum\limits_{\beta = 1}^{N_\text{slip}} \chi^{\alpha\beta} \rho_\text{SSD}^\beta \left|sin \left( n^\alpha , t^\beta \right) \right|  \label{eq:4.2} \\
\end{equation}
\end{subequations}
%
where \begin{math} \chi^{\alpha\beta} \end{math} is the interaction strength between the different slip systems, \begin{math} n^\alpha \end{math} is the slip plane normal and \begin{math} t^\beta \end{math} 
is the line direction in a particular slip system. 

To account for the effects from the other dislocation types, the mean free path is described as,
%
\begin{equation}
\frac{1}{\Lambda_s^\alpha} = K_1 \sqrt{\rho_F^\alpha} + K_2 \sqrt{\rho_{\text{wall,immobile}}^\alpha} + \frac{1}{D} + \frac{K_3}{d_s} \label{eq:5}
\end{equation}
%
where \begin{math} K_1,K_2,K_3 \end{math} are the proportionality constants. 
These constants relate the effect of these particular entities alone on the overall dislocation mean free path. 
Based on the mean free path of the dislocations, the possible initial size of the dislocations cells/sub-grains can be calculated, by assuming,
%
\begin{equation}
d_s = 2r_s = 2\Lambda_s \label{eq:6}
\end{equation}
%
This gives the initial cell/sub-grain size at the first time step. 
For later stage of subgrain growth, the grain growth rate equations should be used for calculations. 
The use of equation \eqref{eq:6} would lead to a very large sub grain size initially (due to low dislocation density). 
But, the further evolution of the subgrain size would depend on the grain size evolution equation dependent on the shrinkage and growth terms.
Mostly, in the initial stages of deformation, the increase in the immobile dislocations is quite rapid \cite{Roters2000}.
Therefore, in the initial stages of the deformation, the sub-grain shrinkage would dominate leading to smaller sub-grains eventually. 

The mobile dislocations in the interior of the dislocation cell/subgrain can get reduced when they interact with the forest dislocations and form a lock leading to immobile dislocations. 
The rate of reduction of mobile dislocations in this manner, can be considered as a process, where a mobile dislocation encounters a forest dislocation as it moves in a particular time step. 
This gives following expression, motivated from the work of Ma and Roters et al. \cite{Ma2004}

\begin{equation}
\dot\rho_{\text{m,lock}}^{\alpha-} = \frac{c_4\left|\dot\gamma^\alpha\right|\sqrt{\rho_F^\alpha}}{b} \label{eq:7}
\end{equation}

The increase in the mobile dislocation density can be described as shown in \cite{Roters2017}

\begin{equation}
\dot\rho_{\text{m}}^{\alpha+} = \frac{\left|\dot\gamma^\alpha\right|}{b\Lambda_s^\alpha} \label{eq:8}
\end{equation}

The other source of reduction of the mobile dislocations is the interaction of these dislocations with the dislocations having opposite signs and within the distance of annihilation. 
This can be calculated based on the approach described by Roters et al. \cite{Roters2000}, leading to equation,

\begin{equation}
\dot\rho_{\text{m,annihilate}}^{\alpha-} = 2\check{d} v_\alpha \rho_m^\alpha = \frac{2\check{d}\left|\dot\gamma^\alpha\right|}{b} \label{eq:9}
\end{equation}
%
where, \begin{math} \check{d} \end{math} is the distance from the dislocation upto which the annihilation can happen. This can be treated as a fitting parameter or a constant as done by Roters et al. in
\cite{Roters2000} and \cite{Roters2017} respectively. As the approach, mentioned by Roters et al. \cite{Roters2000} is followed, it is assumed that there is an equal density of positive and negative dislocations of all types.   

In a similar way, the reduction of mobile dislocation density due to dipole formation can be taken into account. 
Here, the dipole formation due to mobile dislocations in the cell interior and the immobile dislocations in the cell interiors are considered.

\begin{equation}
\dot\rho_{\text{m,dipole}}^{\alpha-} = \frac{2\left(\hat{d} - \check{d}\right)}{b} \left|\dot\gamma^\alpha\right| \rho_{\text{m}}^\alpha + \frac{\left(\hat{d} - \check{d}\right)}{b} \left|\dot\gamma^\alpha\right| \rho_{\text{imm,cell}}^\alpha \label{eq:10}
\end{equation}
%
The mobile dislocations might also interact with the migrating cell/subgrain boundaries. 
It is assumed that the interacting dislocations would settle in the boundaries by locking with the other dislocations present in the boundary. 
This rate of reduction of dislocation densities would be dependent on the relative velocities of mobile dislocations and the LAGB. 
But, this model doesn’t take the signed dislocation densities into account and thus, finding relative velocities is difficult. 
To circumvent this problem, it is assumed that the dislocations of positive and negative signs are equally distributed. 
This introduces factor of \begin{math} \frac{1}{2} \end{math}.
The difference in the velocities of the mobile dislocations and boundary would ensure that some dislocations (which glide in direction opposite to the boundary) would settle in the boundary. 
Not all the interacting dislocations would get caught up in the boundary. 
The ability of the boundary to lock the interacting dislocation would depend on the dislocation density in the boundary. 
Thus, a factor which quantifies this effect is multiplied at the end. 
This can be described by the expression,

% 
\begin{equation}
\dot\rho_{\text{m,LAGB}}^{\alpha-} = \frac{1}{2} \pi d_s \left|\overrightarrow{v}_\text{LAGB} - \overrightarrow{v}_\alpha \right| \times \rho_m^\alpha \times \frac{\rho_{\text{imm,wall}}^\alpha}{\rho_{\text{total}}} \label{eq:11}
\end{equation}
%
The method to calculate the velocity of the sub-grain/cell boundary is mentioned later. 
In the equation above, \begin{math} \rho_{\text{total}}^\alpha \end{math} is the sum of all different dislocations in a slip system being considered in this model. 

%
The increase in the number of immobile dislocations inside the cells is due to immobilization of the mobile dislocations when they interact with the forest dislocations in the cell interiors. 
Thus, it can be correlated to the lock formation rate inside the cell interiors.

%
\begin{equation}
\dot\rho_{\text{imm,cell}}^{\alpha+} = \dot\rho_{\text{m,lock}}^{\alpha-} = \frac{c_4 \left|\dot\gamma^\alpha\right| \sqrt{\rho_F^\alpha}}{b} \label{eq:12}
\end{equation}

%
The immobile dislocations can form dipoles with the mobile dislocations, if the distance between these two types of dislocations is enough to form a dipole 
This approach has been described in \cite{Roters2011}

\begin{subequations}
\begin{equation}
\dot\rho_{\text{imm,cell,dipole}}^{\alpha-} = \frac{\hat{d} - \check{d}}{b} \left|\dot\gamma^\alpha\right| \rho_{\text{imm,cell}}^{\alpha} \label{eq:13}
\end{equation}

\begin{equation}
\dot\rho_{\text{imm,cell,annihilate}}^{\alpha-} = \left(\check{d} v_\alpha \rho_{\text{imm,cell}}^\alpha \right) \rho_m^\alpha = \frac{\check{d} \dot\gamma^\alpha \rho_{\text{imm,cell}}^\alpha}{b} \label{eq:13_1}
\end{equation}
\end{subequations}
%
When the cell/subgrain boundaries (LAGB) migrate, they might interact with the immobile dislocations in the cell interiors and absorb them. 
This would increase the dislocation density inside the LAGB, but reduce the immobile dislocation density in the interior of the cells. 
This can be described in the same way as done in case of equation \eqref{eq:11}
the only difference would be that the velocity of the dislocations will not be taken into account,
%
\begin{equation}
\dot\rho_{\text{imm,cell,LABG}}^{\alpha-} = \pi d_s \left| v_{\text{LABG}} \right| \times \rho_{\text{imm,cell}}^{\alpha} \times \frac{\rho_{\text{imm,wall}}^{\alpha}}{\rho_{\text{total}}^{\alpha}} \label{eq:14}
\end{equation}
%
The other process to reduce the immobile dislocation density is through the climb process of the dislocations. 
The rate equation for this process is given by \cite{Roters2000} ,

\begin{equation}
\dot\rho_{\text{imm,cell,climb}}^{\alpha-} = 2 v_\text{climb} \check{d} {\left(\rho_{\text{imm}}^{\alpha}\right)}^2 \label{eq:15}
\end{equation}

%
where the climb velocity is defined as, 
\begin{math} v_\text{climb} = A' c_j {\left( \frac{\gamma_\text{SFE}}{Gb} \right)}^{2} \frac{D_\text{SD}}{b} \frac{\Omega \sigma_c }{k_B T}\end{math}
as defined in \cite{Eisenlohr2004} and \cite{Argon1981}. 
The calculation of \begin{math} D_{\text{SD}} \end{math} can be considered a little differently. 
Generally, the activation energy for self-diffusion is only considered for lattice self-diffusion. 
But, according to Turunen and Lindroos \cite{Turunen1974} the pipe diffusion of vacancies can also be a contributing factor. 
Therefore, the activation energy can be assumed as \begin{math} E = \frac{E_{\text{SD,lattice}} + E_{\text{SD,pipe}}}{2} \end{math}.
This debate has also been highlighted by experimental data of recovery kinetics of Gottstein \cite{Lu2011} . 
The term \begin{math} \sigma_c \end{math} also needs to be defined carefully for non-dipole configurations. 
\begin{math} \sigma_c \end{math} is generally the sum of dislocation self-stresses and the external stresses. 
Here, the external stresses are neglected and only the dislocation self-stresses are considered. 
Therefore, the \begin{math} \sigma_c = \frac{Gb \sqrt{\rho_{\text{imm}}^\alpha}}{2 \pi \left( 1 - \nu \right)} \end{math} , derivation of which is shown in the appendix A. 
% 


The immobile dislocations in the interior of the cells can lead to formation of new cell boundaries as envisioned by Nes \cite{Nes1997}.
Following a similar approach, it is assumed that a fraction (\begin{math} f \end{math}) of leftover immobile dislocations 
\begin{math} f \left(\dot\rho_{\text{imm,cell}}^{\alpha+} - \dot\rho_{\text{imm,cell}}^{\alpha-} \right) = f \dot\rho_{\text{imm,cell,total}}^{\alpha}\end{math} will form the new cell boundaries. 
Therefore, it is assumed that the cell/subgrain size during deformation is inversely proportional to the leftover immobile dislocation density. 
This can also help in explaining the saturation of sub-grain sizes at higher strains and temperatures as observed by Nes \cite{Nes1997}.
%
The other contributing factor to reduction in dislocation densities is from cross slip, which is common for screw dislocations in FCC metals. 
This mechanism is being considered for reduction of immobile dislocations in the cell interiors and the cell walls. 
It can be assumed that half of the total dislocations are screw dislocations. 
Based on the activation energy for the cross slip based on Friedel Escaig mechanism \cite{Puschl2002}, the probability of cross slip can be used to estimate the annihilation due to cross slip,

\begin{equation}
\dot\rho_{\text{imm,cell,cross}}^{\alpha-} = c_5 \omega b^2 exp\left( - \frac{E_{\text{CS}}}{K_B T} \right) \frac{\rho_{\text{imm,cell}}^2}{4} \label{eq:16}
\end{equation}

where \begin{math} c_5 \end{math} is a proportionality constant, \begin{math} \omega \end{math} is the vibration frequency of atoms, \begin{math} E_{\text{CS}} \end{math} is the activation energy of the cross slip, 
which is dependent on the stacking fault energy and the shear stress on the slip system \cite{Puschl2002}. 
It is assumed that a screw dislocation can slip in the activation area surrounding it. 
When this slipping dislocation encounters another screw dislocation with opposite sign, it will get annihilated. 

The dislocation dipoles also have their separate rate equations. 
The dipole formation rate equations have already been derived in the earlier equations. 
It is assumed that all the dipoles being formed end up in the cell/subgrain walls, as done by Roters \cite{Roters2000}. 
The equations describing the reduction in the dipole dislocation density will be discussed further. 
The reduction in the dipole dislocation density is due to two processes – dipole dissociation annihilation due to thermal effects and dipole dissociation though stress increase.


The increase in the resolved shear stress due to hardening leads to reduction of the distance below which the two dislocations would remain in a stable dipole configuration. 
This contributes to the reduction in the dipole dislocation density and leads to creation of mobile dislocations again. 
For these changes in the dipole density, only the dipoles in the cell interiors are considered. 
It is assumed that the dipoles in the cell walls would get affected only through the thermal effects.

\begin{equation}
\dot\rho_{\text{mob}}^{\alpha-} = \dot\rho_{\text{dipole,cell}}^{\alpha-} = \rho_{\text{cell,dipole}}^{\alpha} \frac{1}{\hat{d} - \check{d}} \frac{d\hat{d}}{dt} \label{eq:17}
\end{equation} 

The thermal annihilation occurs due to climb of dislocations leading to dissociation of the dipoles. 

\begin{equation}
\dot\rho_{\text{dipole,cell}}^{\alpha-} = \rho_{\text{cell,dipole}}^{\alpha} \frac{4 v_{\text{climb}}}{\hat{d} - \check{d}} \label{eq:18}
\end{equation} 

The dislocations in the cell walls would be considered now. 
It is assumed that all the dislocations in the cell walls are immobile due to intense entanglement and the only possible movement is through climb or cross-slip. 
The dislocations in cell walls can undergo following processes: annihilation with dislocation of an opposite value through process of climb or cross-slip, 
increase through absorption of dislocation dipoles and other mobile and immobile dipoles when the GB moves. 
The leftover immobile dislocations inside the cells are assumed to contribute to formation of new LAGBs, through formation of new boundaries. 
Therefore, the increase in dislocation density in the walls should also account for these immobile cell interior dislocations.


The increase in the dislocation density through absorption of different dislocation densities can be described as,
\begin{equation}
\dot\rho_{\text{imm,wall}}^{\alpha+} = \dot\rho_{\text{m,LABG}}^{\alpha-} + \dot\rho_{\text{imm,cell,LABG}}^{\alpha-} + \dot\rho_{\text{dipole}}^{\alpha+} + f \dot\rho_{\text{imm,cell,total}}^{\alpha} \label{eq:19}
\end{equation} 

where \begin{math} \dot\rho_{\text{imm,cell,total}}^{\alpha} \end{math} is the rate of formation of leftover immobile dislocation densities.
It can be described as the difference between the dislocation formation part and dislocation reduction part. 

The reduction in the dislocation density in the cell walls can be described through a climb process as described in \cite{Roters2000},

\begin{subequations}
\begin{equation}
\dot\rho_{\text{imm,wall,climb}}^{\alpha-} = 2v_{\text{climb,wall}} \check{d} \left(\rho_{\text{imm,wall}}^{\alpha}\right)^2 \label{eq:20_1}
\end{equation}

\begin{equation}
\dot\rho_{\text{imm,wall,cross}}^{\alpha-} = c_6 b^2 exp\left( - \frac{E_{\text{CS}}}{K_B T} \right) \frac{\rho_{\text{imm,wall,lock}}^2}{4} \label{eq:20_2}
\end{equation}
\end{subequations}

It is assumed here, that the flow stress in the walls and interior is the same, which leads to a similar value of \begin{math} E_{\text{CS}} \end{math} for cross slip in both locations. 


All these equations together give the total dislocation density evolution for a cell/sub-grain structures. 
The evolution rate for the dislocation densities can help in getting the sub-grain growth equations. 
The excess immobile dislocations in the interiors of the sub-grains contribute to reduction of subgrain sizes. 
Therefore, rate of reduction of the sub-grain size can be described by assuming that the sub-grain size is inversely proportional to left over immobile dislocations. 
This leads to time based sub-grain size reduction rate equation as,

\begin{equation}
\dot{d}_s^- = - \frac{1}{2} K_c \rho_{\text{imm,total}}^{-\frac{3}{2}} {\dot{\rho}}_{\text{imm,cell,total}} \label{eq:21}
\end{equation}

The growth of the sub-grains/cells can be described by a simple grain growth theory. 
The velocity of the sub-grain boundary can be described as \begin{math} v = MF \end{math}, where M is the temperature dependent mobility and F is the driving force. 
It is assumed that the pre-exponential factor in the description of the mobility is constant. 
With deformation, the dislocation density in the cell/subgrain boundaries changes, which, in turn changes the energy of the boundary.

The energy of low angle sub-grain boundaries depends on the misorientation through the Read-Shockley equation \cite{Read1950}, 

\begin{equation}
E = \gamma_o \theta [A - ln \theta] \label{eq:22}
\end{equation}

where \begin{math} E \end{math} is the low angle grain boundary energy, \begin{math} \theta \end{math} is the misorientation of the low angle grain boundary,
\begin{math} \gamma_o = \frac{Gb}{4\pi \left( 1 - \nu \right)} \end{math} and \begin{math} A = 1 + ln \frac{b}{2 \pi r_o} \end{math}

The misorientation is based on the GND density. 
In this model, it is assumed that the dislocations other than the dipole dislocations in the walls, would contribute to misorientations in the cell/subgrain boundaries. 
The GND density  can be approximated as: 
\begin{math} \rho_{\text{GND}} = \rho_{\text{imm,wall}} - \rho_{\text{imm,wall,dipoles}} \end{math}

\begin{equation}
\theta = \frac{\rho_{\text{GND}} ub}{2}  \label{eq:23}
\end{equation}

where, \begin{math} u \end{math} is the unit length. 
Based on this, the rate of increase of the cell/subgrain size can be described as,

\begin{equation}
{\dot{d}}_s^+ = M\left(\frac{2E}{R}\right)  \label{eq:24}
\end{equation}

Therefore, the total change in the size of sub-grain would be,

\begin{equation}
{\dot{d}} = {\dot{d}}_s^- + {\dot{d}}_s^+  \label{eq:24}
\end{equation}

The sub-grain size would affect the mean free path and the passing stress for the dislocations through the equations \eqref{eq:3} and \eqref{eq:5}. 
This would change the dislocation density evolution, leading to interdependence between the sub-grain sizes and dislocation density evolution.


\bibliographystyle{ieeetran}
%can use bibliographystyle{plain} for the most basic things but it doesnt order the citations properly
\bibliography{library}
\end{document}